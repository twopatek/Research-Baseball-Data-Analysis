% Options for packages loaded elsewhere
\PassOptionsToPackage{unicode}{hyperref}
\PassOptionsToPackage{hyphens}{url}
%
\documentclass[
]{article}
\usepackage{amsmath,amssymb}
\usepackage{iftex}
\ifPDFTeX
  \usepackage[T1]{fontenc}
  \usepackage[utf8]{inputenc}
  \usepackage{textcomp} % provide euro and other symbols
\else % if luatex or xetex
  \usepackage{unicode-math} % this also loads fontspec
  \defaultfontfeatures{Scale=MatchLowercase}
  \defaultfontfeatures[\rmfamily]{Ligatures=TeX,Scale=1}
\fi
\usepackage{lmodern}
\ifPDFTeX\else
  % xetex/luatex font selection
\fi
% Use upquote if available, for straight quotes in verbatim environments
\IfFileExists{upquote.sty}{\usepackage{upquote}}{}
\IfFileExists{microtype.sty}{% use microtype if available
  \usepackage[]{microtype}
  \UseMicrotypeSet[protrusion]{basicmath} % disable protrusion for tt fonts
}{}
\makeatletter
\@ifundefined{KOMAClassName}{% if non-KOMA class
  \IfFileExists{parskip.sty}{%
    \usepackage{parskip}
  }{% else
    \setlength{\parindent}{0pt}
    \setlength{\parskip}{6pt plus 2pt minus 1pt}}
}{% if KOMA class
  \KOMAoptions{parskip=half}}
\makeatother
\usepackage{xcolor}
\usepackage[margin=1in]{geometry}
\usepackage{graphicx}
\makeatletter
\newsavebox\pandoc@box
\newcommand*\pandocbounded[1]{% scales image to fit in text height/width
  \sbox\pandoc@box{#1}%
  \Gscale@div\@tempa{\textheight}{\dimexpr\ht\pandoc@box+\dp\pandoc@box\relax}%
  \Gscale@div\@tempb{\linewidth}{\wd\pandoc@box}%
  \ifdim\@tempb\p@<\@tempa\p@\let\@tempa\@tempb\fi% select the smaller of both
  \ifdim\@tempa\p@<\p@\scalebox{\@tempa}{\usebox\pandoc@box}%
  \else\usebox{\pandoc@box}%
  \fi%
}
% Set default figure placement to htbp
\def\fps@figure{htbp}
\makeatother
\setlength{\emergencystretch}{3em} % prevent overfull lines
\providecommand{\tightlist}{%
  \setlength{\itemsep}{0pt}\setlength{\parskip}{0pt}}
\setcounter{secnumdepth}{-\maxdimen} % remove section numbering
\usepackage{bookmark}
\IfFileExists{xurl.sty}{\usepackage{xurl}}{} % add URL line breaks if available
\urlstyle{same}
\hypersetup{
  pdftitle={NCAA Pitching Dashboard Read Me},
  pdfauthor={Matthew Adams},
  hidelinks,
  pdfcreator={LaTeX via pandoc}}

\title{NCAA Pitching Dashboard Read Me}
\author{Matthew Adams}
\date{2025-07-17}

\begin{document}
\maketitle

\section{NCAA Pitching Analysis
Dashboard}\label{ncaa-pitching-analysis-dashboard}

\subsection{Overview}\label{overview}

This project consists of a comprehensive \textbf{Shiny dashboard} built
to analyze, explore, and visualize NCAA Division I baseball
\textbf{pitching data}. It is part of a broader effort to construct and
maintain a \textbf{custom NCAA pitching dataset} that does not currently
exist in a centralized or accessible format.

The app is designed for baseball analysts, researchers, and fans
interested in evaluating historical and team-level pitching performance
across multiple dimensions.

\begin{center}\rule{0.5\linewidth}{0.5pt}\end{center}

\subsection{Live Dashboard Prototype}\label{live-dashboard-prototype}

➡️
\textbf{\href{https://matthew-apps.shinyapps.io/baseball_data_analysis/}{Click
here to explore the NCAA Pitching Analysis Dashboard}}\\
\emph{The app is hosted via
\href{https://www.shinyapps.io}{shinyapps.io}.}

\subsection{License}\label{license}

This project is proprietary and all rights are reserved by Matthew
Adams.

Unauthorized copying, modification, distribution, or commercial use is
strictly prohibited.\\
For collaboration inquiries or licensing requests, contact:
\href{mailto:mdadams626@outlook.com}{\nolinkurl{mdadams626@outlook.com}}.

\subsection{Dashboard Features}\label{dashboard-features}

\subsubsection{Leaderboard Tab}\label{leaderboard-tab}

Quickly explore top pitching performers with built-in filters: -
\textbf{Innings Pitched Slider}: Adjust the minimum and maximum IP
threshold - \textbf{Top 5 SO (Team-Year \& Player-Year)} - \textbf{Top 5
ERA (Team-Year \& Player-Year)} - \textbf{Top 5 WHIP (Team-Year \&
Player-Year)}\\
Displayed using interactive DataTables for easy comparison.

\begin{center}\rule{0.5\linewidth}{0.5pt}\end{center}

\subsubsection{Data Tab}\label{data-tab}

Dive into the raw player-level pitching data: - \textbf{Filter by School
\& Year} (with select/deselect all toggle) - \textbf{Dynamic IP Slider}
for custom filtering - \textbf{Reset Button} to return to default view -
\textbf{Interactive Table} shows all available player records matching
filter criteria

\begin{center}\rule{0.5\linewidth}{0.5pt}\end{center}

\subsubsection{Analysis Tab}\label{analysis-tab}

Split into two panels for deep-dive reporting and plotting:

\paragraph{Report Tab}\label{report-tab}

\begin{itemize}
\tightlist
\item
  Filter data by school
\item
  Dynamically select which numeric metrics (e.g., ERA, WHIP, SO/9) to
  summarize
\item
  Generate grouped \textbf{summary stats} by school and year (mean,
  median, standard deviation)
\item
  Switch between raw data view and summary output
\end{itemize}

\paragraph{Plot Tab}\label{plot-tab}

\begin{itemize}
\tightlist
\item
  Select a school and variable to analyze over time
\item
  Choose display mode:

  \begin{itemize}
  \tightlist
  \item
    \textbf{Raw Only} (mean value by year)
  \item
    \textbf{Summary Only} (mean, median, and SD)
  \item
    \textbf{Both}
  \end{itemize}
\item
  Output is rendered as a grouped \textbf{interactive bar chart} via
  Plotly
\end{itemize}

\subsubsection{Player Ratings Tab}\label{player-ratings-tab}

Comprehensive system for evaluating pitchers across multiple dimensions:
- \textbf{Customizable prior shrinkage factors} (Bayesian smoothing by
IP) - \textbf{User-defined composite rating weights} for ERA, FIP, SO/9,
BB/9, etc. - League-year averages and rescaling to \textbf{0--100 rating
scale} - \textbf{Dynamic recalculation of ratings} with adjustable
inputs - Two views: \textbf{Player Ratings} and full \textbf{Methodology
Table} with raw/shrunk/rescaled stat breakdown - Expanded Info tab with
full explanation of rating logic and design

\begin{center}\rule{0.5\linewidth}{0.5pt}\end{center}

\subsection{Data Structure}\label{data-structure}

Data is loaded from a local directory (\texttt{/Data/}) and cleaned
using the \texttt{janitor} package. Each file typically includes:

\begin{itemize}
\tightlist
\item
  \texttt{school}, \texttt{year}, \texttt{name}, \texttt{ip},
  \texttt{era}, \texttt{whip}, \texttt{bb9}, \texttt{so9}, \texttt{h9},
  \texttt{hr9}, etc.
\item
  Only pitching stats are used. Hitting stats are excluded.
\end{itemize}

\begin{quote}
⚠️ \textbf{Note}: Large-scale data ingestion is ongoing. Full D1
coverage is a long-term goal.
\end{quote}

\begin{center}\rule{0.5\linewidth}{0.5pt}\end{center}

\end{document}
